\section{Conclusions and Future Work}
\label{sec:conclusion}

In this paper, we have conducted extensive studies on system and application behavior using various job placement and routing configurations on dragonfly network topologies. We took a simulation-based approach, utilizing the CODES simulation toolkit and related models for high-fidelity dragonfly simulation, driving the network with three production-representative scientific application traces. We found that, under the prevailing recommendation of random process placement and adaptive routing, the network effectively distributes the workload traffic to achieve a balanced load and strong overall performance, at the cost of impairing jobs with less intensive communication patterns. We identify this phenomenon as the ``bully" effect. On the other hand, contiguous process placement prevents such effects while exacerbating local congestion, though this can be mitigated somewhat through the addition of adaptive routing. Finally, we performed initial experimentation using a mock ``hybrid'' contiguous/random job placement policy, showing the effect of specializing allocation strategies on the resulting communication performance.
We believe the findings and methodology presented in this paper can illuminate a path to a more efficient workload manager for future systems with dragonfly networks. 

%\TODO{more future work here?}
In the future, we plan to explore the intelligent job placement and routing configurations for diversified workloads running on the dragonfly network. The dragonfly connected system could have dedicated job placement and routing configuration based on the characteristics of its workload. We plan to collect workload traces from a production HPC system, and experiment with the simulated network of the system to explore the dedicated optimal job placement and routing configuration. We envision that effectiveness of such new placement and routing polices can be examined in future on a configurable dragonfly system. 

