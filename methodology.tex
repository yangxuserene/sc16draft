

\section{Methodology}
\label{sec: methodology}

Configurable dragonfly networks that allow us to perform the exploration presented in this paper are hard to come by for the time being. Even with access to systems with such networks, job placement and routing policies are part of system configuration, which is impossible for user to make changes at will~\cite{zhou-ipdps-2015}\cite{jain-sc14}\cite{bhatele-sc11}\cite{jokanovic-ipdps-2015}. Therefore, we resort to simulation in our work.


%It is difficult to experiment with concurrently running jobs on HPC systems~\cite{zhou-ipdps-2015}\cite{jain-sc14}\cite{bhatele-sc11}\cite{jokanovic-ipdps-2015}. One reason is that job placement and routing policies are part of system configuration, which is impossible for user to make changes at will. Another reason is that it is unrealistic to reserve the system exclusively to run the same batch of jobs with desired placement and routing configurations and compare the results. The third reason is that configurable dragonfly networks that allow us to perform the exploration presented in this paper are hard to come by for the time being. Therefore, we resort to simulation in our work.

\subsection{Simulation Tool}
\label{sec:simulation-tool}

A simulation toolkit named CODES, Co-Design of Multilayer Exascale Storage Architectures, enables the exploration of simulating different HPC networks with high fidelity and great scalability~\cite{codes}. CODES supports dragonfly~\cite{codes-dragonfly}~\cite{misbah-tpds}, torus~\cite{misbah-pads-2014}~\cite{ning-pads-2011}, and fat-tree~\cite{ning-pads-2015} networks with packet-level high-fidelity simulation. It can drive these models through an MPI simulation layer utilizing DUMPI traces~\cite{sst}. In this work, we perform an in-depth analysis of network performance as well as job interference when different job placement and routing policies are in use on the dragonfly network.

\subsection{Parallel Applications}
\label{sec: application traces}

We choose three parallel applications from the DOE Design Forward Project~\cite{designforwardwebpage}, which can represent a wide array of applications running on leadership-class supercomputers. Specifically, we study the \emph{Algebraic MultiGrid Solver} (AMG), \emph{Geometric Multigrid V-Cycle from Production Elliptic Solver} (MultiGrid) and \emph{Crystal Router MiniApp} (CrystalRouter). 

\textbf{AMG:} The Algebraic MultiGrid Solver is a parallel algebraic multi-grid solver for linear systems arising from problems on unstructured mesh physics packages. It has been derived directly from the BoomerAMG solver developed in the Center for Applied Scientific Computing (CASC) at LLNL~\cite{amg}. 


\textbf{MultiGrid:} The geometric multi-grid v-cycle from the production elliptic solver BoxLib is a software framework for massively parallel block-structured adaptive mesh refinement (AMR) codes~\cite{boxlib}, which are used for structured grid physics packages. 

\textbf{CrystalRouter:} The Crystal Router MiniApp is the communication kernel of the full application Nek5000~\cite{nek5000}, a spectral element CFD application developed at Argonne National Laboratory. It features spectral element multi-grid solvers coupled with a highly scalable, parallel coarse-grid solver that is widely used for projects including ocean current modeling, thermal hydraulics of reactor cores, and spatiotemporal chaos. 




\subsection{System Configuration}
\label{sec: simulation configuration}

The parameters for building the dragonfly network studied in our work are inspired by the model proposed by Kim et. al~\cite{kim-micro}. Our dragonfly network consists of 33 groups, each of which contains eight routers. Each router has four compute nodes attached to it. Overall, there are 264 routers and 1056 compute nodes in the network. The aggregated bandwidth of terminal link,  local and global channels are proportional to those of the Cray Cascade system~\cite{faanes}. In this work, we study both the network performance and job interference when six different job placement and routing policy combinations are in use on the dragonfly network, which are summarized in Table~\ref{tab: placement routing configs}. 

%\footnote{With respect to random placement, we experiment with 50 sets of distinctive allocation generated by random placement. The corresponding experimental results are median chosen, which intended to eliminate the possibility of variation.}

\begin{table}[ht]
\begin{center}
\caption{The notation for different placement and routing configurations} 
\label{tab: placement routing configs}
\begin{tabular}{l c c c }
\toprule % Top horizontal line
\toprule
&\multicolumn{3}{c}{Routing Policies} \\ 
\cmidrule(l){2-4}
Placement Policies & Minimal & Adaptive & Progressive Adaptive\\ % Column names row
\midrule % In-table horizontal line
Contiguous  &  CM   &   CA   &  CPA   \\ % Content row 1
\midrule
Random  &   RM  &   RA   &  RPA   \\ 
\midrule % In-table horizontal line
\bottomrule % Bottom horizontal line
\end{tabular}
\end{center}
\end{table}


Our analysis focuses on the following metrics:
\begin{itemize}

    \item \textbf{Network Traffic:} The traffic refers to the amount of data going through each router. We analyze the traffic on the terminal link, local and global channel of each router. The network reaches optimal performance when the traffic is uniformly distributed and no particular router is over-loaded. 
            
    \item \textbf{Network Saturated Time:} The saturated time refers to the time period when the buffer of a certain port in the router is full. We analyze the saturated time of ports corresponding to terminal link, local and global channel. The saturated time indicates the congestion level of routers. 
    
    \item \textbf{Communication Time:} The communication time of each MPI rank refers to the time it spends in completing all its message exchanges with other ranks. The performance of each application is measured by the communication time.  
\footnote{Since our study focuses on network performance and job interference during communication, the computation time of each MPI rank will not be counted in the experiments.
}

\end{itemize}



\subsection{Workload Summary}
\label{sec:workload summary}

Two sets of parallel workloads are used in this work. Workload~\Rmnum{1} consists of AMG, MultiGrid and CrystalRouter. In Table~\ref{tab:apps-detail}, AMG has the least amount of data transfer, making it the least communication-intensive job in Workload~\Rmnum{1}. Workload~\Rmnum{2} consists of sAMG, MultiGrid and CrystalRouter. sAMG, a synthetic version of AMG, is generated by increasing AMG data transfer amount by 100x. sAMG is the most communication-intensive job in Workload~\Rmnum{2}. 

Workload~\Rmnum{1} is used to identify the ``bully" behavior among concurrently running jobs on the dragonfly network. Workload~\Rmnum{2} is used to verify the results and conclusion obtained from the study of Workload~\Rmnum{1}.

%Table \ref{tab:apps-detail} summarizes the details of each application. It shows the number of MPI ranks, the average data transfer amount of each rank and the total data transfer amount for each application.

\begin{table}[ht]
\begin{center}
\caption{The details of each application.} 
\label{tab:apps-detail}
\begin{tabular}{l c c c }
\toprule % Top horizontal line
\toprule
&\multicolumn{3}{c}{Application Details} \\ 
\cmidrule(l){2-4}
App Name & Num. Rank & Avg. Data/Rank & Total Data\\ % Column names row
\midrule % In-table horizontal line
AMG  &    216 &   0.6MB   &     130MB\\ % Content row 1
\midrule
MultiGrid  &    125 &   5MB   &     625MB\\ 
\midrule
CrystalRouter  &   100  &  35MB    &     3500MB\\ 
\midrule
sAMG  &    216 &   60MB   &     13000MB\\ % Content row 1
\midrule % In-table horizontal line
\bottomrule % Bottom horizontal line
\end{tabular}
\end{center}
\end{table}

