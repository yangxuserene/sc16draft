
\section{Related Work}
\label{sec:related work}

The impact of job placement to both system and application always catches researchers' appetite \cite{dskinner} \cite{abhinav-sc13} \cite{jose-ipdps15}. Skinner et al. identified that network congestion could cause significant performance variability \cite{dskinner}. Bhatele et al. studied the performance variability of a specific application, p3FD, running on different production systems with torus networks \cite{abhinav-sc13}. They noticed that the application performance would keep consistent when it got compact allocation and exclusive network resource. Jokanovic et al. studied the impact of job placement to the workload and claimed that the key to reduce performance variability is to avoid network sharing \cite{jose-ipdps15}. 

Recently, several researchers have investigated the job placement and routing schemes on dragonfly network. Bogdan et al. proposed novel solution for mapping tasks of application that conforms to Nearest Neighbor communication pattern on dragonfly network \cite{hoefler-hpdc14}. Jain et al. conducted a comprehensive analysis of various job placement and routing policies with regard to network link throughput on dragonfly network \cite{jain-sc14}. Their work is based on an analytical model and synthetic workload. Bhatele et al. used simulation to study the performance of synthetic workload under the different task mapping and routing policies on two-level direct networks \cite{bhatele-sc11}.


Our work different from the previous works in the following ways. First, instead of using synthetic workload that generated based on predefined communication patterns, our simulation is driven by real application traces collected from production system. Second, not only we care about the workload performance, but also we pay attention to each individual application in the workload. We found that although random placement coupled with adaptive routing can improve the network performance, the less communication-intensive applications in the workload may suffer performance degradation. Third, with the sophisticated simulation tool CODES, we can not only get applications communication information, but also the insight about the network performance when application is running. The traffic and saturated time of router's each link are valuable information for detailed analysis.


